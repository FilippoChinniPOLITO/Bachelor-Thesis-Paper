\chapter*{\Large \center Abstract}

Machine Learning models have become increasingly more complex, with a large number of hyperparameters to tune. In particular, Neural Networks are currently the most widely used models, but they are also among the most computationally expensive to train.
Hyperparameter optimization (HPO) is the process of identifying the optimal hyperparameters for a given model, and it is a crucial aspect of model success. Additionally, with HPO, the structure, or architecture, of a neural network can be optimized, and thus simplified, reducing its complexity with minimal loss of performance.
After introducing HPO, with its most successful techniques, this thesis presents methodologies to evaluate and compare existing techniques, with the use of a popular HPO library.
Then the thesis proceeds to adapt a popular optimization technique, Particle Swarm Optimization (PSO), to HPO, and compares its performance to the existing techniques.
The study next proposes an experiment on a real-world problem, specifically Semantic Segmentation for Drone Vision, with the objective of optimizing the hyperparameters and the architecture of a complex neural network model on a complex real-world dataset.
The thesis concludes with a global evaluation of the work done, explaining the potential for future work in the field of Hyperparameter Optimization.