%Acronyms and Symbols Lists

\newacronym{ai}{AI}{- Artificial Intelligence}
\newacronym{ml}{ML}{- Machine Learning}
\newacronym{dl}{DL}{- Deep Learning}
\newacronym{nn}{NN}{- Neural Network}
\newacronym{dnn}{DNN}{- Deep Neural Network}
\newacronym{hp}{HP}{- Hyperparameter}
\newacronym{hpo}{HPO}{- Hyperparameter Optimization}
\newacronym{svm}{SVM}{- Support Vector Machine}
\newacronym{bo}{BO}{- Bayesian Optimization}
\newacronym{sgd}{SGD}{- Stochastic Gradient Descent}
\newacronym{pbt}{PBT}{- Population Based Training}
\newacronym{sha}{SHA}{- Successive Halving Algorithm}
\newacronym{asha}{ASHA}{- Asynchronous Successive Halving Algorithm}
\newacronym{bohb}{BOHB}{- Bayesian Optimization and Hyperband}
\newacronym{tpe}{TPE}{- Tree-structured Parzen Estimator}
\newacronym{mab}{MAB}{- Multi-Armed Bandit}
\newacronym{cv}{CV}{- Cross-Validation}
\newacronym{ncv}{NCV}{- Nested Cross-Validation}
\newacronym{cpu}{CPU}{- Central Processing Unit}
\newacronym{gpu}{GPU}{- Graphics Processing Unit}
\newacronym{oo}{OO}{- Object-Oriented}
\newacronym{cma-es}{CMA-ES}{- Covariance Matrix Adaptation Evolution Strategy}
\newacronym{nsgaii}{NSGAII}{- Non-dominated Sorting Genetic Algorithm II}
\newacronym{qmc}{QMC}{- Quasi-Monte Carlo}
\newacronym{psot}{PSO}{- Particle Swarm Optimization}
\newacronym{ga-pso}{GA-PSO}{- Hybrid of Genetic Algorithm and Particle Swarm Optimization}
\newacronym{epso}{EPSO}{- Hybrid of Evolutionary Programming and Particle Swarm Optimization}
\newacronym{apso}{APSO}{- Adaptive Particle Swarm Optimization}
\newacronym{mopso}{MOPSO}{- Multi-Objective Particle Swarm Optimization}
\newacronym{dpso}{DPSO}{- Discrete Particle Swarm Optimization}
\newacronym{mlp}{MLP}{- Multi-Layer Perceptron}
\newacronym{RAM}{RAM}{- Random Access Memory}
\newacronym{SSD}{SSD}{- Solid State Drive}

% \newglossaryentry{chi}{
%   name={$\textbf{$\chi$}$},
%   description={ - Space of all possible Hyperparameters Configurations}
% }
% \newglossaryentry{X-bold}{
%   name={$\textbf{X}$},
%   description={ - Search Space}
% }
% \newglossaryentry{X}
% {
%   name={$X$},
%   description={ - Hyperparameter Configuration (or Candidate Solution, or Trial)}
% }
% \newglossaryentry{f}
% {
%   name={$f$},
%   description={ - Objective Function}
% }




% Chapter 1

\chapter{Introduction}

The introduction should present the topic of the thesis to specify the purpose and importance of the work. It is common practice to have a summary at the beginning of each chapter, like this one.

\section{Aims and Objectives}

The introduction should familiarize the reader with the problem to be addressed and describes:
\begin{itemize}
    \item The current level of knowledge (with references) as a basis for the work;
    \item The knowledge gap the study wants to fill;
    \item Outline of the objectives without anticipating results.
\end{itemize}
And possibly:
\begin{itemize}
	\item The chosen approach;
	\item The structure of the thesis.
\end{itemize}

\section{Overview of the Thesis}

The rest of this thesis is structured as follows: 
\begin{itemize}
    \item Chapter~2 discusses background knowledge\ldots
    \item \ldots
\end{itemize}
